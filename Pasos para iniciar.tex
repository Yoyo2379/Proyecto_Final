\documentclass[12pt,a4paper]{article}
\usepackage[utf8]{inputenc}
\usepackage[spanish]{babel}
\usepackage{geometry}
\usepackage{xcolor}
\usepackage{fancyhdr}
\usepackage{listings}
\usepackage{enumitem}
\usepackage{amsmath}
\usepackage{amsfonts}
\usepackage{amssymb}
\usepackage{graphicx}
\usepackage{hyperref}
\usepackage{tcolorbox}

\geometry{margin=2.5cm}
\pagestyle{fancy}
\fancyhf{}
\fancyhead[L]{\textbf{Sistema de Gestión de Tareas y Proyectos}}
\fancyhead[R]{\thepage}
\fancyfoot[C]{\textit{Angular 21 + Spring Boot 3.2}}


\definecolor{codegreen}{rgb}{0,0.6,0}
\definecolor{codegray}{rgb}{0.5,0.5,0.5}
\definecolor{codepurple}{rgb}{0.58,0,0.82}
\definecolor{backcolour}{rgb}{0.95,0.95,0.92}

\lstdefinestyle{mystyle}{
    backgroundcolor=\color{backcolour},   
    commentstyle=\color{codegreen},
    keywordstyle=\color{magenta},
    numberstyle=\tiny\color{codegray},
    stringstyle=\color{codepurple},
    basicstyle=\ttfamily\footnotesize,
    breakatwhitespace=false,         
    breaklines=true,                 
    captionpos=b,                    
    keepspaces=true,                 
    numbers=left,                    
    numbersep=5pt,                  
    showspaces=false,                
    showstringspaces=false,
    showtabs=false,                  
    tabsize=2
}
\lstset{style=mystyle}

\newtcolorbox{infobox}[1]{
    colback=blue!5!white,
    colframe=blue!75!black,
    title=#1,
    fonttitle=\bfseries
}

\newtcolorbox{successbox}[1]{
    colback=green!5!white,
    colframe=green!75!black,
    title=#1,
    fonttitle=\bfseries
}

\newtcolorbox{warningbox}[1]{
    colback=orange!5!white,
    colframe=orange!75!black,
    title=#1,
    fonttitle=\bfseries
}

\newtcolorbox{errorbox}[1]{
    colback=red!5!white,
    colframe=red!75!black,
    title=#1,
    fonttitle=\bfseries
}

\begin{document}

\begin{center}
    {\Huge \textbf{Pasos para iniciar}}\\[0.5cm]
    {\Large Sistema de Gestión de Tareas y Proyectos}\\[0.3cm]
    {\large \textbf{SISTEMA UNIFICADO} - Angular 21 + Spring Boot 3.2 + MySQL}\\[0.3cm]
    {\normalsize \textit{Un solo comando, una sola aplicación}}\\[0.5cm]
    \rule{\linewidth}{0.5mm}
\end{center}

\vspace{1cm}

\section{Requisitos Previos}

\begin{successbox}{Sistema Simplificado}
El sistema ahora está completamente unificado. Solo necesitas:
\begin{itemize}[label=\textbullet]
    \item \textbf{Java 17 o superior} (REQUERIDO)
    \item \textbf{MySQL Server 8.0} (REQUERIDO)
    \item \textbf{JAR ejecutable} (ya compilado)
\end{itemize}
\end{successbox}

\begin{infobox}{Ya NO necesitas}
\begin{itemize}[label=\textbullet]
    \item Node.js (frontend ya integrado)
    \item npm (frontend ya compilado)
    \item Maven (JAR ya compilado)
    \item Dos terminales (solo una aplicación)
\end{itemize}
\end{infobox}

\section{Paso 1: Verificar MySQL Server}

\subsection{Opción A: Verificar desde Servicios de Windows}

\begin{lstlisting}[language=bash, caption=Abrir Servicios de Windows]
services.msc
\end{lstlisting}

\begin{enumerate}
    \item Buscar ``MySQL80'' o ``MySQL'' en la lista de servicios
    \item El estado debe ser ``En ejecución'' (Running)
    \item Si no está corriendo, hacer clic derecho $\rightarrow$ ``Iniciar''
\end{enumerate}

\subsection{Opción B: Verificar desde línea de comandos}

\begin{lstlisting}[language=bash, caption=Verificar puerto MySQL]
netstat -an | findstr :3306
\end{lstlisting}

\begin{successbox}{Resultado Esperado}
Si MySQL está corriendo, debería ver:
\begin{verbatim}
TCP    0.0.0.0:3306           0.0.0.0:0              LISTENING
\end{verbatim}
\end{successbox}

\subsection{Opción C: Iniciar MySQL si no está corriendo}

\begin{lstlisting}[language=bash, caption=Iniciar servicio MySQL]
net start MySQL80
\end{lstlisting}

\section{Paso 2: Preparar el Entorno}

\subsection{Navegar al directorio del proyecto}

\begin{lstlisting}[language=bash, caption=Cambiar al directorio del proyecto]
cd "C:\Users\oswal\Desktop\Proyecto monroy"
\end{lstlisting}

\subsection{Verificar estructura del proyecto}

\begin{lstlisting}[language=bash, caption=Listar contenido del directorio]
dir
\end{lstlisting}

\begin{infobox}{Estructura Esperada}
Debería ver las siguientes carpetas:
\begin{itemize}
    \item \texttt{fronted/} - Aplicación Angular
    \item \texttt{Proyecto\_final/} - Aplicación Spring Boot
\end{itemize}
\end{infobox}

\section{Paso 3: Ejecutar el Sistema Unificado}

\subsection{Comando único}

\begin{lstlisting}[language=bash, caption=Iniciar aplicación completa]
cd Proyecto_final
java -jar target/gestion-tareas-1.0.0.jar
\end{lstlisting}

\begin{successbox}{Una sola aplicación}
Este comando inicia:
\begin{itemize}
    \item Backend Spring Boot (puerto 8080)
    \item Frontend Angular (integrado)
    \item API REST completa
    \item Documentación Swagger
    \item Interfaz web completa
\end{itemize}
\end{successbox}

\subsection{Verificar inicio exitoso}

\begin{successbox}{Señales de Éxito}
El sistema está funcionando cuando vea:
\begin{itemize}
    \item \texttt{Started GestionTareasApplication in X.XXX seconds}
    \item \texttt{Tomcat started on port 8080 (http)}
    \item No aparecen errores de conexión a MySQL
    \item La consola se queda ``esperando'' (no regresa al prompt)
\end{itemize}
\end{successbox}

\begin{warningbox}{Tiempo de Inicio}
El sistema puede tardar entre 8-15 segundos en iniciar completamente.
\end{warningbox}

\section{Paso 4: Verificación Final}

\subsection{URLs del Sistema Unificado}

\begin{center}
\begin{tabular}{|l|l|}
\hline
\textbf{Componente} & \textbf{URL} \\
\hline
\textbf{Aplicación Web Completa} & \url{http://localhost:8080} \\
\hline
Documentación Swagger & \url{http://localhost:8080/swagger-ui.html} \\
\hline
API REST & \url{http://localhost:8080/api} \\
\hline
\end{tabular}
\end{center}

\begin{infobox}{Todo en un solo puerto}
Ya no necesitas recordar múltiples URLs. Todo está en:
\textbf{http://localhost:8080}
\end{infobox}

\subsection{Checklist de Verificación}

\begin{enumerate}[label=\textbf{\arabic*.}]
    \item \textbf{MySQL corriendo:} Puerto 3306 en uso
    \item \textbf{Sistema unificado:} Puerto 8080 responde
    \item \textbf{Aplicación web:} http://localhost:8080 carga correctamente
    \item \textbf{Dashboard:} Muestra estadísticas reales de tareas
    \item \textbf{CRUD completo:} Proyectos y tareas funcionando
    \item \textbf{Swagger:} Documentación API accesible
\end{enumerate}

\section{Solución de Problemas}

\subsection{Si MySQL no inicia}

\begin{errorbox}{Error de MySQL}
\begin{lstlisting}[language=bash]
# Iniciar servicio manualmente
net start MySQL80

# O desde Servicios de Windows
services.msc
\end{lstlisting}
\end{errorbox}

\subsection{Si el Sistema no inicia}

\begin{errorbox}{Error del Sistema Unificado}
\begin{lstlisting}[language=bash]
# Verificar Java
java -version

# Si el JAR no existe, recompilar
cd Proyecto_final
mvn clean package -DskipTests

# Ejecutar nuevamente
java -jar target/gestion-tareas-1.0.0.jar
\end{lstlisting}
\end{errorbox}

\subsection{Si hay problemas con la aplicación web}

\begin{errorbox}{Error de Aplicación Web}
\begin{lstlisting}[language=bash]
# Verificar que el sistema esté corriendo
curl http://localhost:8080/api/projects

# Si la API funciona pero la web no:
# 1. Limpiar caché del navegador
# 2. Probar en modo incógnito
# 3. Verificar consola del navegador (F12)
\end{lstlisting}
\end{errorbox}

\section{Comandos Rápidos}

\begin{successbox}{Inicio Ultra Rápido}
Para iniciar el sistema completo en 2 comandos:

\textbf{Paso 1 - Verificar MySQL:}
\begin{lstlisting}[language=bash]
netstat -an | findstr :3306
\end{lstlisting}

\textbf{Paso 2 - Ejecutar Sistema:}
\begin{lstlisting}[language=bash]
cd Proyecto_final && java -jar target/gestion-tareas-1.0.0.jar
\end{lstlisting}

\textbf{Listo! Abrir navegador en:} \url{http://localhost:8080}
\end{successbox}

\section{Ventajas del Sistema Unificado}

\begin{successbox}{Beneficios de la Unificación}
\begin{itemize}
    \item \textbf{Un solo comando} para iniciar todo el sistema
    \item \textbf{Una sola URL} para acceder a toda la aplicación
    \item \textbf{Sin dependencias externas} (Node.js, npm no requeridos)
    \item \textbf{Despliegue simplificado} - solo copiar un archivo JAR
    \item \textbf{Menos errores} - no hay problemas de integración
    \item \textbf{Más rápido} - no hay compilación en tiempo real
    \item \textbf{Profesional} - listo para producción
\end{itemize}
\end{successbox}

\begin{infobox}{Comparación: Antes vs Ahora}
\textbf{Antes (Sistema Separado):}
\begin{itemize}
    \item 2 terminales, 2 comandos, 2 puertos
    \item Frontend: http://localhost:4200
    \item Backend: http://localhost:8080
    \item Dependencias: Java + Node.js + npm + Maven
\end{itemize}

\textbf{Ahora (Sistema Unificado):}
\begin{itemize}
    \item 1 terminal, 1 comando, 1 puerto
    \item Todo: http://localhost:8080
    \item Dependencias: Solo Java + MySQL
\end{itemize}
\end{infobox}

\section{Información del Sistema}

\begin{center}
\begin{tabular}{|l|l|}
\hline
\textbf{Componente} & \textbf{Versión/Tecnología} \\
\hline
\textbf{Sistema} & \textbf{Unificado JAR Ejecutable} \\
\hline
Frontend & Angular 21.0.2 (integrado) \\
\hline
Backend & Spring Boot 3.2.0 \\
\hline
Base de Datos & MySQL 8.0 \\
\hline
Java & OpenJDK 17+ \\
\hline
UI Framework & Angular Material \\
\hline
Documentación API & Swagger/OpenAPI 3.0 \\
\hline
Despliegue & Un solo archivo JAR \\
\hline
\end{tabular}
\end{center}

\vfill

\begin{center}
\rule{\linewidth}{0.5mm}\\
\textbf{SISTEMA COMPLETAMENTE UNIFICADO}\\
\textit{Un comando, una aplicación, máxima simplicidad}\\[0.3cm]
\textbf{Listo para demostración profesional}\\[0.2cm]
{\large \texttt{java -jar target/gestion-tareas-1.0.0.jar}}\\
{\normalsize \textit{http://localhost:8080}}
\end{center}

\end{document}