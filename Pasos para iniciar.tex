\documentclass[12pt,a4paper]{article}
\usepackage[utf8]{inputenc}
\usepackage[spanish]{babel}
\usepackage{geometry}
\usepackage{xcolor}
\usepackage{fancyhdr}
\usepackage{listings}
\usepackage{enumitem}
\usepackage{amsmath}
\usepackage{amsfonts}
\usepackage{amssymb}
\usepackage{graphicx}
\usepackage{hyperref}
\usepackage{tcolorbox}

% Configuración de página
\geometry{margin=2.5cm}
\pagestyle{fancy}
\fancyhf{}
\fancyhead[L]{\textbf{Sistema de Gestión de Tareas y Proyectos}}
\fancyhead[R]{\thepage}
\fancyfoot[C]{\textit{Angular 21 + Spring Boot 3.2}}

% Configuración de colores
\definecolor{codegreen}{rgb}{0,0.6,0}
\definecolor{codegray}{rgb}{0.5,0.5,0.5}
\definecolor{codepurple}{rgb}{0.58,0,0.82}
\definecolor{backcolour}{rgb}{0.95,0.95,0.92}

% Configuración de código
\lstdefinestyle{mystyle}{
    backgroundcolor=\color{backcolour},   
    commentstyle=\color{codegreen},
    keywordstyle=\color{magenta},
    numberstyle=\tiny\color{codegray},
    stringstyle=\color{codepurple},
    basicstyle=\ttfamily\footnotesize,
    breakatwhitespace=false,         
    breaklines=true,                 
    captionpos=b,                    
    keepspaces=true,                 
    numbers=left,                    
    numbersep=5pt,                  
    showspaces=false,                
    showstringspaces=false,
    showtabs=false,                  
    tabsize=2
}
\lstset{style=mystyle}

% Configuración de cajas
\newtcolorbox{infobox}[1]{
    colback=blue!5!white,
    colframe=blue!75!black,
    title=#1,
    fonttitle=\bfseries
}

\newtcolorbox{successbox}[1]{
    colback=green!5!white,
    colframe=green!75!black,
    title=#1,
    fonttitle=\bfseries
}

\newtcolorbox{warningbox}[1]{
    colback=orange!5!white,
    colframe=orange!75!black,
    title=#1,
    fonttitle=\bfseries
}

\newtcolorbox{errorbox}[1]{
    colback=red!5!white,
    colframe=red!75!black,
    title=#1,
    fonttitle=\bfseries
}

\begin{document}

% Título
\begin{center}
    {\Huge \textbf{Pasos para iniciar}}\\[0.5cm]
    {\Large Sistema de Gestión de Tareas y Proyectos}\\[0.3cm]
    {\large Angular 21 + Spring Boot 3.2 + MySQL}\\[0.5cm]
    \rule{\linewidth}{0.5mm}
\end{center}

\vspace{1cm}

\section{Requisitos Previos}

\begin{infobox}{Verificar Instalaciones}
Antes de iniciar el sistema, asegúrese de tener instalado:
\begin{itemize}[label=\textbullet]
    \item \textbf{Java 17 o superior}
    \item \textbf{Node.js 18 o superior}
    \item \textbf{MySQL Server 8.0}
    \item \textbf{Maven 3.6 o superior}
    \item \textbf{npm 9 o superior}
\end{itemize}
\end{infobox}

\section{Paso 1: Verificar MySQL Server}

\subsection{Opción A: Verificar desde Servicios de Windows}

\begin{lstlisting}[language=bash, caption=Abrir Servicios de Windows]
services.msc
\end{lstlisting}

\begin{enumerate}
    \item Buscar ``MySQL80'' o ``MySQL'' en la lista de servicios
    \item El estado debe ser ``En ejecución'' (Running)
    \item Si no está corriendo, hacer clic derecho $\rightarrow$ ``Iniciar''
\end{enumerate}

\subsection{Opción B: Verificar desde línea de comandos}

\begin{lstlisting}[language=bash, caption=Verificar puerto MySQL]
netstat -an | findstr :3306
\end{lstlisting}

\begin{successbox}{Resultado Esperado}
Si MySQL está corriendo, debería ver:
\begin{verbatim}
TCP    0.0.0.0:3306           0.0.0.0:0              LISTENING
\end{verbatim}
\end{successbox}

\subsection{Opción C: Iniciar MySQL si no está corriendo}

\begin{lstlisting}[language=bash, caption=Iniciar servicio MySQL]
net start MySQL80
\end{lstlisting}

\section{Paso 2: Preparar el Entorno}

\subsection{Navegar al directorio del proyecto}

\begin{lstlisting}[language=bash, caption=Cambiar al directorio del proyecto]
cd "C:\Users\oswal\Desktop\Proyecto monroy"
\end{lstlisting}

\subsection{Verificar estructura del proyecto}

\begin{lstlisting}[language=bash, caption=Listar contenido del directorio]
dir
\end{lstlisting}

\begin{infobox}{Estructura Esperada}
Debería ver las siguientes carpetas:
\begin{itemize}
    \item \texttt{fronted/} - Aplicación Angular
    \item \texttt{Proyecto\_final/} - Aplicación Spring Boot
\end{itemize}
\end{infobox}

\section{Paso 3: Iniciar el Backend (Spring Boot)}

\subsection{Abrir primera terminal}

\begin{lstlisting}[language=bash, caption=Navegar al backend e iniciar]
cd Proyecto_final
mvn spring-boot:run
\end{lstlisting}

\subsection{Verificar inicio exitoso}

\begin{successbox}{Señales de Éxito}
El backend está funcionando cuando vea:
\begin{itemize}
    \item \texttt{Started GestionTareasApplication in X.XXX seconds}
    \item \texttt{Tomcat started on port(s): 8080}
    \item No aparecen errores de conexión a MySQL
    \item La consola se queda ``esperando'' (no regresa al prompt)
\end{itemize}
\end{successbox}

\begin{warningbox}{Tiempo de Inicio}
El backend puede tardar entre 30-60 segundos en iniciar completamente.
\end{warningbox}

\section{Paso 4: Verificar Backend Funcionando}

\subsection{Probar API REST}

\begin{lstlisting}[language=bash, caption=Verificar endpoint de proyectos]
curl http://localhost:8080/api/projects
\end{lstlisting}

\subsection{Verificar Swagger UI}

Abrir en navegador: \url{http://localhost:8080/swagger-ui.html}

\begin{successbox}{Verificación Exitosa}
Si el backend funciona correctamente:
\begin{itemize}
    \item El comando curl devuelve JSON con proyectos
    \item Swagger UI carga la documentación de la API
    \item No hay errores en la consola del backend
\end{itemize}
\end{successbox}

\section{Paso 5: Iniciar el Frontend (Angular)}

\subsection{Abrir segunda terminal}

\begin{lstlisting}[language=bash, caption=Navegar al frontend e iniciar]
cd "C:\Users\oswal\Desktop\Proyecto monroy"
cd fronted
npm start
\end{lstlisting}

\subsection{Verificar compilación}

\begin{successbox}{Señales de Éxito}
El frontend está funcionando cuando vea:
\begin{itemize}
    \item \texttt{Local: http://localhost:4200/}
    \item \texttt{webpack compiled successfully}
    \item Se abre automáticamente el navegador
    \item No hay errores de compilación
\end{itemize}
\end{successbox}

\begin{warningbox}{Tiempo de Compilación}
El frontend puede tardar entre 30-45 segundos en compilar completamente.
\end{warningbox}

\section{Paso 6: Verificación Final}

\subsection{URLs del Sistema}

\begin{center}
\begin{tabular}{|l|l|}
\hline
\textbf{Componente} & \textbf{URL} \\
\hline
Aplicación Web & \url{http://localhost:4200} \\
\hline
Documentación Swagger & \url{http://localhost:8080/swagger-ui.html} \\
\hline
API REST & \url{http://localhost:8080/api} \\
\hline
\end{tabular}
\end{center}

\subsection{Checklist de Verificación}

\begin{enumerate}[label=\textbf{\arabic*.}]
    \item \textbf{MySQL corriendo:} Puerto 3306 en uso
    \item \textbf{Backend funcionando:} Puerto 8080 responde
    \item \textbf{Frontend funcionando:} Puerto 4200 carga la aplicación
    \item \textbf{Integración:} Dashboard muestra estadísticas reales
    \item \textbf{Datos:} Listas de proyectos y tareas cargan información
\end{enumerate}

\section{Solución de Problemas}

\subsection{Si MySQL no inicia}

\begin{errorbox}{Error de MySQL}
\begin{lstlisting}[language=bash]
# Iniciar servicio manualmente
net start MySQL80

# O desde Servicios de Windows
services.msc
\end{lstlisting}
\end{errorbox}

\subsection{Si el Backend no inicia}

\begin{errorbox}{Error de Backend}
\begin{lstlisting}[language=bash]
# Verificar Java
java -version

# Limpiar y reinstalar dependencias
cd Proyecto_final
mvn clean install
mvn spring-boot:run
\end{lstlisting}
\end{errorbox}

\subsection{Si el Frontend no inicia}

\begin{errorbox}{Error de Frontend}
\begin{lstlisting}[language=bash]
# Verificar Node.js y npm
node -v
npm -v

# Limpiar node_modules y reinstalar
cd fronted
rmdir /s node_modules
npm install
npm start
\end{lstlisting}
\end{errorbox}

\section{Comandos Rápidos}

\begin{infobox}{Inicio Rápido}
Para iniciar el sistema completo en 3 comandos:

\textbf{Terminal 1 - Backend:}
\begin{lstlisting}[language=bash]
cd Proyecto_final && mvn spring-boot:run
\end{lstlisting}

\textbf{Terminal 2 - Frontend:}
\begin{lstlisting}[language=bash]
cd fronted && npm start
\end{lstlisting}

\textbf{Verificación:}
\begin{lstlisting}[language=bash]
curl http://localhost:8080/api/projects
\end{lstlisting}
\end{infobox}

\section{Información del Sistema}

\begin{center}
\begin{tabular}{|l|l|}
\hline
\textbf{Componente} & \textbf{Versión/Tecnología} \\
\hline
Frontend & Angular 21.0.2 \\
\hline
Backend & Spring Boot 3.2.0 \\
\hline
Base de Datos & MySQL 8.0 \\
\hline
Java & OpenJDK 17 \\
\hline
Node.js & 18+ \\
\hline
UI Framework & Angular Material \\
\hline
Documentación API & Swagger/OpenAPI 3.0 \\
\hline
\end{tabular}
\end{center}

\vfill

\begin{center}
\rule{\linewidth}{0.5mm}\\
\textit{Sistema listo para demostración y evaluación}\\
\textbf{¡Éxito garantizado!}
\end{center}

\end{document}